When dealing with image matting, we can construct a graph such that each pixel is a vertex and adjacent pixels share an edge. Such a graph is naturally undirected and non-looping. However, the bulk of image matting is determining appropriate weights for the edges, so we cannot immediately construct a Graph Laplacian. Instead, we use a variant of the Graph Laplacian called the \textbf{Matting Laplacian} \cite{levin08}. Although identical in construction, our $D$ and $W$ must be produced through a different vertex \textbf{affinity function} than simple adjacency. The choice of this affinity function is open, but we use the form shown in \cite{levin08}, which is outlined in \textbf{Techniques}.
\\\\
In general, we can solve for our values of $\alpha$ by minimizing the quadratic form
\[J(\alpha)=\alpha^T L\alpha\]
\textbf{I will explain why this is, in the final draft.} One of the key properties of a matting Laplacian is that it is positive semi-definite, so can be minimized to a non-trivial solution by solving a system of linear equations. The majority of this paper is devoted to methods for efficiently solving this system of equations for any given Matting Laplacian.