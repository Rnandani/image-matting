I here outline the Matting Laplacian (and derivation) performed in \cite{levin08}. Although many alternate (and often simpler) affinity functions are documented, this one is both common and rigorous. This solution makes a few small simplifying assumptions. The first is that the foreground and background images are locally smooth. This leads to sudden changes in the image color being attributed to sudden changes in $\alpha$, a reasonable expectation. The second is that the image is greyscale, which significantly simplifies calculations and can be worked around later. Given this second assumption, greyscale values will throughout this paper be treated as bounded real numbers.
\\\\
From the smoothness assumption, we can locally treat our values of $\alpha$ as a linear function, conditioned on known values for $F$ and $B$, the foreground and background images. Taking $a=\frac{1}{F-B}$ and $b=-\frac{B}{F-B}$, we define our matte by
\[\alpha_i=aI_i+b\]
for all values of $i$ in a small section of the image. These values of $a$ and $b$ can be justified by noting that $I_i\in[B_i,F_i]$, and the chosen $a$ and $b$ produce $\alpha_i:\,[B_i,F_i]\to[0,1]$, as expected.
\\\\
Solving for $F$ and $B$ is now equivalent to solving for $a$ and $b$. This is done by minimizing a cost function that is chosen in \cite{levin08} to be
\[J(\alpha,a,b)=
	\sum_{j\in I}\left(\sum_{i\in w_j}\left(
		\alpha_i-a_jI_i-b_j\right)^2+\epsilon
		a_j^2\right)\]
with $w_j$ a small box around the pixel $j$. The last term of the sum serves to minimize $a_j$ where possible, and acts to both smooth the matte as a whole, and to resolve convergence where results might otherwise be indeterminate (e.g. a box where the image color is constant). Since the minimal values in each box are determinate, and since they overlap at nearby pixels, this cost function allows local changes to propagate through their neighbors. Cost minimization is solved in \cite{levin08} as follows.

\newtheorem{matlapthm}{Theorem}
\begin{matlapthm}
We define $J(\alpha)$ to be $\min\limits_{a,b} J(\alpha,a,b)$. Then
\[J(\alpha)=\alpha^{T}L\alpha\]
where $L$ is an $N\times N$ matrix with
\[L_{i,j}=\sum_{k|(i,j)\in w_k}\left(
	\delta_{ij}-\frac{1}{|w_k|}\left[
	1+\frac{1}{\frac{\epsilon}{|w_k|}+\sigma_k^2}
	(I_i-\mu_k)(I_j-\mu_k)
\right]\right)\]
where $\delta_{ij}$ is the Kronecker delta ($1$ if $i=j$, $0$ otherwise), $\mu_k$ and $\sigma_k$ are the mean and std. dev. of the color in box $w_k$, and $|w_k|$ is the number of pixels in $w_k$. Note that our sum is over each small box (usually of fixed size) that contains both pixels $i$ and $j$, so points far apart from each other produce a value of zero.
\end{matlapthm}
\begin{proof}
For each $w_k$, we define the $(|w_k|+1)\times2$ matrix $G_k$ where the first $|w_k|$ rows are $[I_i,1]$ for each $i\in w_k$, and the last row is $[\sqrt{\epsilon},0]$. Next we define the $(|w_k|+1)\times1$ matrix $\bar{\alpha}_k$ where the first $i$ rows are the values of $\alpha_i$ for $i\in w_k$ and the last is $0$. We can then rewrite $J(\alpha,a,b)$ as
\[J(\alpha,a,b)=\sum_k\left\Vert
G_k\left[\begin{array}{cc}
a_k \\ b_k
\end{array}\right]
 - \bar{\alpha}_k
\right\Vert^2\]
We want to find the optimal values of $a$ and $b$, $a_k^{*},b_k^{*}$ for each $k$ in our window, but by least squares minimization this is now simply
\[(a_k^*,b_k^*)=\text{argmin}\,
\left\Vert
G_k\left[\begin{array}{cc}
a_k \\ b_k
\end{array}\right]
 - \bar{\alpha}_k
\right\Vert=(G_k^{T}G_k)^{-1}G_k^{T}\bar{\alpha}_k\]
We now define $\bar{G}_k=I-G_k(G_k^TG_k)^{-1}G_k^T$, and can rewrite
\[J(\alpha)=
	\sum_k\bar{\alpha}_k^T\bar{G}_k^T
	      \bar{G}_k\bar{\alpha}_k\]
Note that in this form, we need only minimize $J$ in one variable rather than $3$. Finally, we expand our middle terms, $\bar{G}_k^T \bar{G}_k$, and find by algebra that
\[\bar{G}_k^T \bar{G}_k=\left(
	\delta_{ij}-\frac{1}{|w_k|}\left[
	1+\frac{1}{\frac{\epsilon}{|w_k|}+\sigma_k^2}
	(I_i-\mu_k)(I_j-\mu_k)
\right]\right)\]
\textbf{(For my final draft, I will show the steps to produce this equality)}. Plugging this back into our formula for $J(\alpha)$ and rewriting our sum as matrix composition, we have our desired equation.
\end{proof}
$L$ is our matting laplacian. When looked at in a slightly different way, it is actually just the graph laplacian with a different affinity function, as stated previously. We define our adjacency matrix, $W$, by weights rather than by simple edge existence, so that for $i\neq j$, $W_{ij}=L_{ij}$, and $W_{ii}=0$. Then we define our degree matrix accordingly as a diagonal matrix with
\[D_{ii}=\sum_j W_{ij}\]
Now, just as before, $L=D-W$!