We have a non-looping, unweighted, and undirected graph. We take $v_i$ to be an indexing of its vertices.
We first define the \textbf{degree matrix} of our graph, $D$, to be a diagonal matrix where $D_{ii}$ is the degree of $v_i$ (the number of connecting edges).
Next, we define the \textbf{adjacency matrix} of our graph, $W$, a symmetric matrix where $W_{ij}=1$ if and only if there is a direct edge between $v_i$ and $v_j$. Note that since our graph is non-looping, $W_{ii}=0$.
The \textbf{Graph Laplacian}, $L$, of our graph is now defined as $L=D-W$. Diagonal elements then indicate how strongly connected each vertex is to the rest of the graph, and off-diagonal elements are the negation of how strongly connected specific vertices are to each other.