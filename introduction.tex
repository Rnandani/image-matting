\subsection{Image Matting}
Image matting is an extensive field, but in this paper I focus on foreground and background extraction. Given an image $I$, we wish to produce a \textbf{matte} (a trasparency value $\alpha$ at each pixel, with $\alpha_{i}\in[0,1]$) such that at each pixel $i$ we can deconstruct the color value $I_{i}$ into a sum of two samples, one from a foreground color $F_{i}$ and one from a background color $B_{i}$. To produce the color value at each pixel $i$ in our original image, we then take
\[I_{i}=\alpha_{i}F_{i}+(1-\alpha_{i})B_{i}\]
This problem is constrained with a ``sketch" by the user, often called a \textbf{prior}, indicating regions of the image which are known to be either in the foreground or in the background. With enough such constraints, the problem can be sufficiently defined such that an accurate and useful matte is produced.
\subsection{Topics Covered}
\textit{Scalable Matting: A Sub-linear Approach}, by Philip G. Lee and Ying Wu, documents the steps of several modern image matting techniques, and compares them both analytically and practically. Lee and Wu provide a few different methods for each step of the matting algorithm, bit in this paper I focus in on specific examples. I give brief explanations of most algorithms tested, and focus in particular on the Matting Laplacian from \cite{levin08} and the V-Cycle algorithm detailed in \cite{briggs87} and \cite{bramble93}. I then give previews of a few of the applications of this method, before concluding with some experiments performed with my own implementation.