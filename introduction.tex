\subsection{Image Matting}
Image matting is an extensive field, but in this paper I focus on foreground and background extraction. Given an image $I$, we wish to produce a matte (an $\alpha$ value at each pixel, $\alpha_{i}\in[0,1]$) such that at each pixel $i$ we can deconstruct the color value $I_{i}$ into a sum of samples from a foreground color $F_{i}$ and a background color $B_{i}$. To produce the color value in our original image, we then take
\[I_{i}=\alpha_{i}F_{i}+(1-\alpha_{i})B_{i}\]
This is an important, but difficult algorithmic problem in image processing. Direct solving is typically infeasible, so many approximate solutions have been attempted.
\subsection{Topics Covered}
\textit{Scalable Matting: A Sub-linear Approach}, by Philip G. Lee and Ying Wu, documents the steps of several cutting edge image matting techniques, and compares them both analytically and practically. Although several different methods are explored for most of the steps necessary, in this paper I focus in on specific examples. I give brief explanations of most algorithms tested, and focus in particular on the Matting Laplacian from \cite{levin08} and the V-Cycle algorithm detailed in \cite{briggs87} and \cite{bramble93}. I then give previews of a few of the applications of this theory, before concluding with some experiments performed with my own implementation (\textbf{or at least, this is what will be in the final draft!}).