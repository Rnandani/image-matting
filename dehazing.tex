Image dehazing and its similarities to image matting are explored in detail by \cite{he11}. The basic problem is that, in images taken over some distance, floating particles and moisture can lead to dulling of color for more distant objects. This can significantly reduce the quality of the image taken. Dehazing attempts to determine how much this ``haze" has altered the color at each pixel of an image, and to restore the color to what it would have been if unobscured. The general statement of the problem is to find a solution of the form
\[I_i = J_it_i+A_i(1-t_i)\]
As in image matting, $I$ is the color value at each pixel of the original image, and $t$ is a transparency value at each pixel. In the case of dehazing, $t$ is referred to as the \textbf{medium transmission}, specifying how much of the light from the original source reaches the camera without being scattered by ambient particles. $J$ is the color of the desired image at each pixel, referred to as the \textbf{scene radiance}, while $A$ is global atmospheric light.
\\\\
A prior for this problem is produced by noting the existence of a so-called \textbf{dark channel}. \cite{he11} observes from a set of haze-free daytime images that natural images almost always have at least one color channel (red, green, or blue) close to zero at every pixel. In contrast, the ambient light in images with haze increases every pixel's color value uniformly, preventing the existence of a dark channel where significant haze occurs.